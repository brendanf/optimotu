\usetikzlibrary{patterns, intersections, math, positioning}

% use the 7-class colorbrewer "setdark2" palette (skipping color 6 because it's too light)
\definecolor{g1}{HTML}{1b9e77}
\definecolor{g2}{HTML}{d95f02}
\definecolor{g1s1}{HTML}{7570b3}
\definecolor{g1s2}{HTML}{e7298a}
\definecolor{g2s1}{HTML}{66a61e}
\definecolor{g1s3}{HTML}{a6761d}

% pseudotaxa are all gray
\definecolor{g1s4}{HTML}{bbbbbb}
\definecolor{g2s2}{HTML}{bbbbbb}
\definecolor{g3}{HTML}{bbbbbb}
\definecolor{g3s1}{HTML}{bbbbbb}
\definecolor{g3s2}{HTML}{bbbbbb}

\definecolor{unknown}{HTML}{aaaaaa}

%define the family clustering radius
\def\famrad{35}
% define the genus clustering radius
\def\genrad{23}
% define the species clustering radius in genus g1
\def\spradA{13}
% define the species clustering radius in genus g2
\def\spradB{8}

% define the radius of an ASV symbol
\def\asvrad{1.5}

\def\known{known}
\def\denovo{denovo}
\def\reffirst{ref1}
\def\refsecond{ref2}

% define ASVs
% each entry is x/y/genus/species/radius/type
\newcommand{\asvs}{%
  38.181816/215.75/g1/g1s1/\spradA/known,%
%  33.409092/217.22728/g1/g1s1/\spradA/known,%
%  37.613636/219.61363/g1/g1s1/\spradA/known,%
%  16.136364/230.97728/g1/g1s1/\spradA/known,%
  15.681818/237.56818/g1/g1s1/\spradA/known,%
  23.977272/245.63637/g1/g1s1/\spradA/ref1,%
  30.795454/227/g1/g1s1/\spradA/ref1,%
  15.113636/266.54544/g1/g1s2/\spradA/known,%
%  15.681818/269.15909/g1/g1s2/\spradA/known,%
  23.977272/267.22726/g1/g1s3/\spradA/known,%
  96/243/g2/g2s1/\spradB/ref1,%
  98.75/210.29544/g2/g2s2/\spradB/denovo,%
  120/207.11363/g2/g2s1/\spradB/known,%
  %118.181816/210.97728/g2/g2s1/\spradB/known,%
  121.13636/219.04546/g2/g2s1/\spradB/known,%
  99/232/g2/g2s1/\spradB/known,%
  103/255/g2/g2s1/\spradB/ref2,%
  56.704544/278.59091/g1/g1s4/\spradA/denovo,%
  58/289.11365/g1/g1s4/\spradA/denovo,%
  45.56818/285.29544/g1/g1s4/\spradA/denovo,%
  30.136364/296.84091/g1/g1s4/\spradA/denovo,%
  70.11364/295.06818/g1/g1s4/\spradA/denovo,%
  145/290/g3/g3s1/\spradB/denovo,%
  155/290/g3/g3s1/\spradB/denovo,%
  120/304/g3/g3s2/\spradB/denovo,%
  132/306/g3/g3s2/\spradB/denovo,%
  151/282/g3/g3s1/\spradB/denovo%
}

\begin{tikzpicture}[x=1mm, y=1mm]

% define styles for objects
% ASVs are represented as small circles
% the fill color will be defined separately each time it is used, and the
% outline will always be a darker version of the same color
\tikzstyle{asv}=[circle, draw=black!50, fill=black!50, inner sep=0pt]

% different linkages
\tikzstyle{core}=[line width=0.5mm]
\tikzstyle{reffirst}=[line width=0.5mm, densely dashed]
\tikzstyle{refsecond}=[line width=0.5mm, loosely dashed]
\tikzstyle{denovo}=[line width=0.5mm, dotted]

% define the genus of each asv in a list
\def\asvgenus{
  g1, g1, g1, g1, g1, g1, g1, g1, g1, g1,
  g2, g2, g2, g2, g2, g2, g2, g1, g1, g1,
  g1, g1, g2
}

% define the known species of each asv in a list
\def\asvknown{
  g1s1, g1s1, g1s1, g1s1, g1s1, unknown, unknown, g1s2, g1s2, g1s2,
  unknown, unknown, g2s1, g2s1, g2s1, unknown, unknown, unknown, unknown, unknown,
  unknown, unknown, unknown
}

% draw family cluster as outline
\foreach \x / \y in \asvs {
  \draw[line width=0.6mm, draw=none] (\x, \y) circle (\famrad);
}

% clear interior of family cluster
% \foreach \x / \y in \asvs {
%   \fill[white] (\x, \y) circle (\famrad);
% }

% draw genus cluster outlines with thicker lines
% \foreach \x / \y / \color in \asvs {
%   \draw[color=\color!60, line width=0.6mm] (\x, \y) circle (\genrad);
% }

%shade genus clusters with a very light version of the color
% \foreach \x / \y / \color in \asvs {
%   \fill[color=\color!20] (\x, \y) circle (\genrad);
% }

% draw species cluster outlines with thinner lines
% \foreach \x / \y / \genus / \species / \radius [count = \i] in \asvs {
%   \begin{scope}
%     \path[name path=self, save path=\self] (\x, \y) circle (\radius);
%     \foreach \xb / \yb / \genusb / \speciesb / \radiusb [count = \j] in \asvs {
%       \ifnum\i>\j
%         %check if the two circles are the same species
%         \ifx\species\speciesb
%         \else
%         %\pgfmathsetmacro{\dist}{sqrt((\x - \xb)^2 + (\y - \yb)^2)}
%         %\ifdim\dist pt<2*\radius pt
%         \path[name path=other] (\xb, \yb) circle (\radiusb);
%           \clip [name intersections={of = self and other, name = i, total=\t}]
%             \ifnum\t>1
%               (i-1) -- (i-2) -- ([turn] 0:\radiusb) -- ([turn] 90:2*\radiusb) -- ([turn] 90:4*\radiusb) -- ([turn] 90:2*\radiusb) --cycle;
%
%
%             \fi;
%         \fi
%       \fi
%     }
%       \draw[\species, line width=0.6mm] (\x, \y) circle (\radius);
%   \end{scope}}

% shade species clusters with a light version of the color
% when there are two species within the same radius of each other, we need to
% calculate the line between them and set a clip path to avoid shading the
% entire circle
% \foreach \x / \y / \genus / \species / \radius [count = \i] in \asvs {
%   \pgfinterruptboundingbox
%   \begin{scope}
%     \path[name path=self, save path=\self] (\x, \y) circle (\radius);
%     \foreach \xb / \yb / \genusb / \speciesb / \radiusb [count = \j] in \asvs {
%       \ifnum\i>\j
%         %check if the two circles are the same species
%         \ifx\species\speciesb
%         \else
%         %\pgfmathsetmacro{\dist}{sqrt((\x - \xb)^2 + (\y - \yb)^2)}
%         %\ifdim\dist pt<2*\radius pt
%         \path[name path=other] (\xb, \yb) circle (\radiusb);
%           \clip [name intersections={of = self and other, name = i, total=\t}]
%             \ifnum\t>1
%               (i-1) -- (i-2) -- ([turn] 0:\radiusb) -- ([turn] 90:2*\radiusb) -- ([turn] 90:4*\radiusb) -- ([turn] 90:2*\radiusb) --cycle;
%
%
%             \fi;
%         \fi
%       \fi
%     }
%       \fill[\species!40] (\x, \y) circle (\radius);
%   \end{scope}
%   \endpgfinterruptboundingbox
% }

% draw linkage between ASVs
% \foreach \x / \y / \genus / \species / \radius / \type [count = \i] in \asvs {
%   \begin{scope}
%     \foreach \xb / \yb / \genusb / \speciesb / \radiusb / \typeb [count = \j] in \asvs {
%       \ifnum\i>\j
%         %check the distance between the ASVs
%         \pgfmathparse{notgreater(veclen(\x - \xb, \y - \yb), \radius + \radiusb)}
%         \ifnum\pgfmathresult=1
%           \ifx\species\speciesb
%             \ifx\type\denovo
%               \draw[\species, denovo] (\x, \y) -- (\xb, \yb);
%             \else
%               \ifx\type\known
%                 \ifx\typeb\known
%                   \draw[\species, core] (\x, \y) -- (\xb, \yb);
%                 \else
%                   \draw[\species, reffirst] (\x, \y) -- (\xb, \yb);
%                 \fi
%               \else
%                 \ifx\typeb\known
%                   \draw[\species, reffirst] (\x, \y) -- (\xb, \yb);
%                 \else
%                   \ifx\type\refsecond
%                     \ifx\typeb\reffirst
%                       \draw[\species, refsecond] (\x, \y) -- (\xb, \yb);
%                     \fi
%                   \else
%                     \ifx\type\reffirst
%                       \ifx\typeb\refsecond
%                         \draw[\species, refsecond] (\x, \y) -- (\xb, \yb);
%                       \fi
%                     \fi
%                   \fi
%                 \fi
%               \fi
%             \fi
%           \else
%             \ifx\type\known
%               \ifx\typeb\known
%                 \draw[red, core] (\x, \y) -- (\xb, \yb)
%                 node[pos=0.5,sloped] {\sffamily\bfseries{X}};
%               \else
%                 \draw[red, reffirst] (\x, \y) -- (\xb, \yb)
%                 node[pos=0.5,sloped] {\sffamily\bfseries{X}};
%               \fi
%             \else
%               \draw[red, reffirst] (\x, \y) -- (\xb, \yb)
%               node[pos=0.5,sloped] {\sffamily\bfseries{X}};
%             \fi
%           \fi
%         \else
%           \ifx\species\speciesb
%             \ifx\type\known
%               \ifx\typeb\known
%                 \draw[\species, core] (\x, \y) -- (\xb, \yb);
%               \fi
%             \fi
%           \fi
%         \fi
%       \fi
%     }
%   \end{scope}
% }

% shade de-novo clustered ASVs using a very light grey with a dotted pattern
%\foreach \x / \y / \genus / \species / \radius / \type in \asvs {
%  \ifx\type\denovo
%    \fill[pattern=dots, pattern color=\species] (\x, \y) circle (\radius);
%  \fi
%}

% shade second.stage reference clustered ASVs using a gray version of their color
%\foreach \x / \y / \genus / \species / \radius / \type in \asvs {
  %\ifx\type\refsecond
%    \fill[pattern=crosshatch, pattern color=\species!80] (\x, \y) circle (\radius);
%  \fi
%}
% shade first.stage reference clustered ASVs using a light version of their color
%\foreach \x / \y / \genus / \species / \radius / \type in \asvs {
%  \ifx\type\reffirst
%    \fill[\species!60] (\x, \y) circle (\radius);
%    \fill[pattern=north east lines, pattern color=\species!80] (\x, \y) circle (\radius);
%  \fi
%}
% shade known clustered ASVs using their color
%\foreach \x / \y / \genus / \species / \radius / \type in \asvs {
%  \ifx\type\known
%    \fill[\species!60] (\x, \y) circle (\radius);
%  \fi
%}

% draw ASV points
\foreach \x / \y / \genus / \species / \radius / \type in \asvs {
  % \ifx\type\known
  %   \draw[\species!50!black, fill=\species] (\x, \y) circle (\asvrad);
  % \else
    \draw[black, fill=white] (\x, \y) circle (\asvrad);
  % \fi
}

% draw family cluster label with big text
% \node[align=center,node font=\Large\bfseries] at (65, 245) {F};

% draw genus cluster labels with medium text
% \node[align=center,node font=\large\bfseries,color=g1!70] at (43, 263) {G1};
% \node[align=center,node font=\large\bfseries,color=g2!70] at (87, 221) {G2};
% \node[align=center,node font=\large\bfseries,color=unknown!70] at (125, 287) {PG3};

% draw species cluster labels with small text
% \node[align=center,node font=\bfseries,color=g1s1] at (28, 236) {S1};
% \node[align=center,node font=\bfseries,color=g1s2] at (10, 271) {S2};
% \node[align=center,node font=\bfseries,color=g1s3] at (30, 272) {S3};
% \node[align=center,node font=\bfseries,color=g1s4] at (48, 292) {PS4};
% \node[align=center,node font=\bfseries,color=g2s1] at (113, 228) {S5};
% \node[align=center,node font=\bfseries,color=g2s2] at (96, 214) {PS6};
% \node[align=center,node font=\bfseries,color=g3s1] at (150, 293) {PS7};
% \node[align=center,node font=\bfseries,color=g3s2] at (119, 308) {PS8};

% draw legend
% \draw [core] (158,240) -- (173,240) node[right] {Cluster core};
% \draw [reffirst] (158,230) -- (173,230) node[right] {Closed ref. (1st iter.)};
% \draw [refsecond] (158,220) -- (173,220) node[right] {Closed ref. (2nd iter.)};
% \draw [denovo] (158,210) -- (173,210) node[right] {\emph{De novo}};
% \draw [core, red] (158,202) -- (173,202) node[pos=0.5,sloped] {\sffamily\bfseries{X}};
% \draw [reffirst, red] (158,198) -- (173,198) node[pos=0.5,sloped] {\sffamily\bfseries{X}} node[color = black, anchor=west] at ++(0, 2) {Illegal linkage};

% color swatches for known species
% \draw [g1s1!50!black, fill=g1s1] (159,190) circle (\asvrad);
% \draw [g1s2!50!black, fill=g1s2] (163,190) circle (\asvrad);
% \draw [g1s3!50!black, fill=g1s3] (167,190) circle (\asvrad);
\draw [draw=none] (171,190) circle (\asvrad) node[right=2mm] {\phantom{Seq. identified to species}};
\draw [black, fill=white] (171,180) circle (\asvrad) node[right=2mm, color = black] {Sequence};

% Letters to point out specific situations
% \node[draw, fill=white, node font=\Large\bfseries] (A) at (10, 290) {A};
% \draw (A) -- (19, 269);
% \node[draw, fill=white, node font=\Large\bfseries] (B) at (110, 240) {B};
% \draw (B) -- (109, 227);
% \node[draw, fill=white, node font=\Large\bfseries] (C) at (0, 250) {C};
% \draw (C) -- (17, 255);
% \node[draw, fill=white, node font=\Large\bfseries] (D) at (85, 260) {D};
% \draw (D) -- (98, 249);


\end{tikzpicture}
